\documentclass[12pt]{amsart}

\usepackage[dvips]{graphics}
\usepackage[brazil]{babel}
\usepackage[utf8]{inputenc}
\usepackage{amsfonts, amsmath, amsthm, amssymb} 
\usepackage{mathrsfs}
\usepackage{graphicx}
\usepackage{color}
\usepackage{hyperref}
\usepackage[all]{xy}

\setlength{\textwidth}{16cm} %largura do texto
\setlength{\textheight}{22.5cm} %altura do texto
\setlength{\parskip}{2mm} %espaço entre parágrafos
\setlength{\topmargin}{-1cm} %
\setlength{\oddsidemargin}{0.5cm}
\setlength{\evensidemargin}{0.5cm}
% se necessário incluir outros pacotes e comandos.



\begin{document}
\pretolerance10000 %% evita separação de sílabas em final de linhas
\thispagestyle{empty}

\begin{center}
{\bf Universidade Federal do Espírito Santo - campus de Alegre}
\vspace*{0.2cm}\\
{\bf III Semana Acadêmica de Matemática }
\vspace*{0.2cm}\\
\textbf{Matemática e práticas educativas: por uma abordagem problematiza!}
\vspace*{0.2cm}\\
{\bf 13 a 16 de Novembro}
\end{center}



\vspace*{0.8cm}

\begin{center}
{\large \textsc{Título}}\\
\vspace{0.8cm}
Autor 1$^{1}$\\
Autor 2$^{2}$\\ % se necessário adicionar mais autores.
{\small $^{1}$Instituição do autor 1, e-mail do autor 1}\\
{\small $^{2}$Instituição do autor 2, e-mail do autor 2}

\end{center}


\vspace*{0.8cm}
\linespread{1.3} 



\noindent O texto do resumo deve ser escrito nesse espaço. O resumo deve ter no \textbf{MÁXIMO} uma página,  com as referências já incluídas, caso seja do interesse do autor apresentá-las.


\bigskip
\bigskip
\noindent\textsc{Palavras-chave:} primeira palavra; segunda palavra; terceira palavra.






\vspace{0.3cm} %ajustar de tal maneira que fique em apenas uma página.

\begin{thebibliography}{xx}


\bibitem{n1} {\sc N. Autor 1, N. Autor2}, {\it título do livro}, Cidade, Editora, (ano).

\bibitem{n2} {\sc N. Autor 1, N. Autor2}, {\it New Two-Line Arrays Representing PartitionsTítulo do artigo}, Nome da revista, {\bf volume}, (ano), pagina inicial - pagina final.

\bibitem{n4} {\sc N. Autor}, {\it Título da tese ou dissertação}, Dissertação de Mestrado (ou Tese de douturado), Instituição, (ano).


\end{thebibliography}


\end{document}